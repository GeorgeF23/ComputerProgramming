%%%%%%%%%%%%%%%%%%%%%%%%%%%%%%%%%%%%%%%%%%%%%%%%%%%%%%%
% MatPlotLib and Random Cheat Sheet
%
% Edited by Michelle Cristina de Sousa Baltazar
%
% http://matplotlib.org/api/pyplot_summary.html
% http://matplotlib.org/users/pyplot_tutorial.html
%
%%%%%%%%%%%%%%%%%%%%%%%%%%%%%%%%%%%%%%%%%%%%%%%%%%%%%%%

\documentclass[a4paper]{article}
\usepackage[landscape]{geometry}
\usepackage{url}
\usepackage{multicol}
\usepackage{amsmath}
\usepackage{amsfonts}
\usepackage{tikz}
\usetikzlibrary{decorations.pathmorphing}
\usepackage{amsmath,amssymb}
\usepackage{hyperref}

\usepackage{colortbl}
\usepackage{xcolor}
\usepackage{mathtools}
\usepackage{amsmath,amssymb}
\usepackage{enumitem}

% Mihnea
\usepackage{textcomp} % \textquotesingle: Racket: '(1 2 3)
\usepackage{couriers}
\usepackage{listings}
\lstset{
	numbers			= left,
	numberstyle		= \tiny,
	numbersep       = 5pt,
	captionpos		= b,
	breaklines		= true,
	basicstyle		= \ttfamily\footnotesize,
	tabsize			= 4,
	escapeinside	= {~}{~},
}
\lstdefinelanguage{Racket}{
  morekeywords=[1]{define, define-syntax, define-macro, lambda, define-stream, stream-lambda},
  morekeywords=[2]{begin, call-with-current-continuation, call/cc,
    call-with-input-file, call-with-output-file, case, cond,
    do, else, for-each, if,
    let*, let, let-syntax, letrec, letrec-syntax,
    let-values, let*-values,
    and, or, not, delay, force,
    quasiquote, quote, unquote, unquote-splicing,
    map, fold, syntax, syntax-rules, eval, environment },
  morekeywords=[3]{import, export},
  alsodigit=!\$\%&*+-./:<=>?@^_~,
  sensitive=true,
  morecomment=[l]{;},
  morecomment=[s]{\#|}{|\#},
  morestring=[b]",
  basicstyle=\footnotesize\ttfamily,
  keywordstyle=\color[rgb]{0,.3,.7},
  commentstyle=\color[rgb]{0.133,0.545,0.133},
  stringstyle={\color[rgb]{0.75,0.49,0.07}},
  upquote=true,
  breaklines=true,
  breakatwhitespace=true,
  literate=*{`}{{`}}{1}
}

\title{Racket}
\usepackage[brazilian]{babel}
\usepackage[utf8]{inputenc}

\advance\topmargin-1.0in
\advance\textheight3in
\advance\textwidth3in
\advance\oddsidemargin-1.5in
\advance\evensidemargin-1.5in
\parindent0pt
\parskip1pt
\newcommand{\hr}{\centerline{\rule{3.5in}{1pt}}}
%\colorbox[HTML]{e4e4e4}{\makebox[\textwidth-2\fboxsep][l]{texto}
\begin{document}

\begin{center}{\huge{\textbf{Laborator 4. Programare modulară. Funcţii în limbajul C. Dezvoltarea algoritmilor folosind funcţii }}}
% {\large Laborator}
\end{center}
\begin{multicols*}{3}

\tikzstyle{mybox} = [draw=black, fill=white, very thick,
    rectangle, rounded corners, inner sep=10pt, inner ysep=10pt]
\tikzstyle{fancytitle} =[fill=black, text=white, font=\bfseries]

% Mihnea
\tikzstyle{mybox_code} = [mybox, draw = orange, fill=sandybrown]
\tikzstyle{fancytitle_code} = [fancytitle, fill = orange]

\definecolor{almond}{rgb}{0.94, 0.87, 0.8}
\definecolor{apricot}{rgb}{0.98, 0.81, 0.69}
\definecolor{atomictangerine}{rgb}{1.0, 0.6, 0.4}
\definecolor{sandybrown}{rgb}{0.96, 0.64, 0.38}
\definecolor{buff}{rgb}{0.94, 0.86, 0.51}

\definecolor{persianred}{rgb}{0.8, 0.2, 0.2}
\definecolor{papayawhip}{rgb}{1.0, 0.94, 0.84}
\tikzstyle{mybox_persianred} = [mybox, draw = persianred, fill=papayawhip]
\tikzstyle{fancytitle_persianred} = [fancytitle, fill = persianred]

\definecolor{whitesmoke}{rgb}{0.96, 0.96, 0.96}
\definecolor{wenge}{rgb}{0.39, 0.33, 0.32}
\tikzstyle{mybox_blue} = [mybox, draw = wenge, fill=whitesmoke]
\tikzstyle{fancytitle_blue} = [fancytitle, fill = wenge]

\definecolor{cerise}{rgb}{0.87, 0.19, 0.39}
\definecolor{mistyrose}{rgb}{1.0, 0.89, 0.88}
\tikzstyle{mybox_cerise} = [mybox, draw = cerise, fill=mistyrose]
\tikzstyle{fancytitle_cerise} = [fancytitle, fill = cerise]

\definecolor{pinegreen}{rgb}{0.0, 0.47, 0.44}
\definecolor{bubbles}{rgb}{0.91, 1.0, 1.0}
\tikzstyle{mybox_pinegreen} = [mybox, draw = pinegreen, fill=bubbles]
\tikzstyle{fancytitle_pinegreen} = [fancytitle, fill = pinegreen]

\definecolor{cream}{rgb}{1.0, 0.99, 0.82}
\definecolor{mikadoyellow}{rgb}{1.0, 0.77, 0.05}
\tikzstyle{mybox_mikadoyellow} = [mybox, draw = mikadoyellow, fill=cream]
\tikzstyle{fancytitle_mikadoyellow} = [fancytitle, fill = mikadoyellow]

\definecolor{cornsilk}{rgb}{1.0, 0.97, 0.86}
\tikzstyle{mybox_orange} = [mybox, draw = orange, fill=cornsilk]
\tikzstyle{fancytitle_orange} = [fancytitle, fill = orange]

\definecolor{aliceblue}{rgb}{0.94, 0.97, 1.0}
\definecolor{seagreen}{rgb}{0.18, 0.55, 0.34}
\tikzstyle{mybox_seagreen} = [mybox, draw = seagreen, fill=aliceblue]
\tikzstyle{fancytitle_seagreen} = [fancytitle, fill = seagreen]

\definecolor{jazzberryjam}{rgb}{0.65, 0.04, 0.37}
\definecolor{almond}{rgb}{0.94, 0.87, 0.8}
\tikzstyle{mybox_jazzberryjam} = [mybox, draw = jazzberryjam, fill=almond]
\tikzstyle{fancytitle_jazzberryjam} = [fancytitle, fill = jazzberryjam]

\definecolor{amaranth}{rgb}{0.9, 0.17, 0.31}
\definecolor{bisque}{rgb}{1.0, 0.89, 0.77}
\tikzstyle{mybox_amaranth} = [mybox, draw = amaranth, fill=bisque]
\tikzstyle{fancytitle_amaranth} = [fancytitle, fill = amaranth]

\definecolor{carminered}{rgb}{1.0, 0.0, 0.22}
\definecolor{blanchedalmond}{rgb}{1.0, 0.92, 0.8}
\tikzstyle{mybox_carminered} = [mybox, draw = amaranth, fill=blanchedalmond]
\tikzstyle{fancytitle_carminered} = [fancytitle, fill = carminered]

\definecolor{midnightgreen}{rgb}{0.0, 0.29, 0.33}
\definecolor{lavendermist}{rgb}{0.9, 0.9, 0.98}
\tikzstyle{mybox_midnightgreen} = [mybox, draw = midnightgreen, fill=lavendermist]
\tikzstyle{fancytitle_midnightgreen} = [fancytitle, fill = midnightgreen]

\definecolor{indigo}{rgb}{0.29, 0.0, 0.51}
\definecolor{isabelline}{rgb}{0.96, 0.94, 0.93}
\tikzstyle{mybox_indigo} = [mybox, draw = indigo, fill=isabelline]
\tikzstyle{fancytitle_indigo} = [fancytitle, fill = indigo]

\definecolor{russet}{rgb}{0.5, 0.27, 0.11}
\definecolor{ivory}{rgb}{1.0, 1.0, 0.94}
\tikzstyle{mybox_russet} = [mybox, draw = russet, fill=ivory]
\tikzstyle{fancytitle_russet} = [fancytitle, fill = russet]

\definecolor{neongreen}{rgb}{0.22, 0.88, 0.08}
\definecolor{splashedwhite}{rgb}{1.0, 0.99, 1.0}
\tikzstyle{mybox_neongreen} = [mybox, draw = neongreen, fill=splashedwhite]
\tikzstyle{fancytitle_neongreen} = [fancytitle, fill = neongreen]

%---------------------------------------------------------------------------------

\begin{tikzpicture}
\node [mybox_persianred] (box){%
    \begin{minipage}{0.3\textwidth}
\textbf
Funcţiile împart taskuri complexe în bucăţi mici mai uşor de înţeles şi de programat.

Crearea funcţiilor trebuie să se bazeze pe următoarele principii: claritate, lizibilitate, uşurinţă în întreţinere, reutilizabilitate.

Antetul unei funcții este:
\begin{lstlisting}[language=Haskell, numbers=none]
tip_returnat nume_functie (tip_param1 nume_param1 , tip_param2 nume_param2, ...); 
\end{lstlisting}
	\end{minipage}
};

\node[fancytitle_persianred, right=10pt] at (box.north west) {Antetul funcției };
\end{tikzpicture}

\begin{tikzpicture}
\node [mybox_persianred] (box){%
    \begin{minipage}{0.3\textwidth}
    	

\begin{lstlisting}[language=Haskell, numbers=none]
tip_returnat nume_functie(tip_param1 nume_param1, tip_param2 nume_param2, ...) {
  declaratii de variabile si instructiuni;
 
  return expresie;
} 
\end{lstlisting}

\textbf Limbajul C permite separarea declaraţiei unei funcţii de definiţia acesteia (codul care o implementează). Pentru ca funcţia să poată fi folosită, este obligatorie doar declararea acesteia înainte de codul care o apelează. 

Definiţia poate apărea mai departe în fişierul sursă, sau chiar într-un alt fişier sursă sau bibliotecă.

\textbf
Tipul returnat de o funcţie poate fi orice tip standard sau definit de utilizator, inclusiv tipul void (care înseamnă că funcția nu returnează nimic).

Orice funcţie care întoare un rezultat trebuie să conţină instrucţiunea:


\begin{lstlisting}[language=Haskell, numbers=none]
 return expression; 

int min(int x, int y); //declarare

int min(int x, int y) { //definire
  if (x < y) {
    return x;
  }
 
  return y;
}
\end{lstlisting}

	\end{minipage}
};

\node[fancytitle_seagreen, right=10pt] at (box.north west) {Definirea funcției};
\end{tikzpicture}

\begin{tikzpicture}
\node [mybox_persianred] (box){%
    \begin{minipage}{0.3\textwidth}

\textbf
Apelul unei funcţii se face specificând parametrii efectivi (parametrii care apar în declararea funcţiei se numesc parametri formali).
\begin{lstlisting}[language=Haskell, numbers=none]
int main() {
  int a, b, minimum;
  //...........
  x = 2;
  y = 5;
  minimum = min(x, 4);
  printf("Minimul dintre %d si 4 este: %d", x, minimum);
  printf("Minimul dintre %d si %d este: %d", x, y, min(x, y));
}
\end{lstlisting}
	\end{minipage}
};

\node[fancytitle_jazzberryjam, right=10pt] at (box.north west) {Apelul funcțiilor};
\end{tikzpicture}

\begin{tikzpicture}
\node [mybox_amaranth] (box){%
    \begin{minipage}{0.3\textwidth}
\textbf
{Transmiterea parametrilor prin valoare}

Funcţia va lucra cu o copie a variabilei pe care a primit-o şi orice modificare din cadrul funcţiei va opera asupra aceste copii. La sfârşitul execuţiei funcţiei, copia va fi distrusă şi astfel se va pierde orice modificare efectuată.

\begin{lstlisting}[language=Haskell, numbers=none]
min(x, 4);  // se face o copie lui x
\end{lstlisting}
    \end{minipage}
};

\node[fancytitle_amaranth, right=10pt] at (box.north west) {Transmiterea parametrilor};
\end{tikzpicture}

\begin{tikzpicture}
\node [mybox_orange] (box){%
    \begin{minipage}{0.3\textwidth}
\textbf
O funcţie poate să apeleze la rândul ei alte funcţii. Dacă o funcţie se apelează pe sine însăşi, atunci funcţia este recursivă. Pentru a evita un număr infinit de apeluri recursive, trebuie ca funcţia să includă în corpul ei o {condiţie de oprire}, astfel ca, la un moment dat, recurenţa să se oprească şi să se revină succesiv din apeluri.
\begin{lstlisting}[language=Haskell, numbers=none]

Calculul recursiv al factorialului
int fact(int n) {
  if (n == 0) {
    return 1;
  } else {
    return n * fact(n - 1);
  }
}

sau

int fact(int n) {
  return (n >= 1) ? n * fact(n - 1) : 1;
}
\end{lstlisting}


    \end{minipage}
};

\node[fancytitle_orange, right=10pt] at (box.north west) {Funcții recursive};
\end{tikzpicture}

\begin{tikzpicture}
\node [mybox_orange] (box){%
    \begin{minipage}{0.3\textwidth}
\textbf
Orice program C conţine cel puţin o funcţie, şi anume cea principală, numită main(). Aceasta are un format special de definire:
\begin{lstlisting}[language=Haskell, numbers=none]
int main(int argc, char *argv[])
{
    // some code
    return 0;
}
\end{lstlisting}
\textbf
Primul parametru, argc, reprezintă numărul de argumente primite de către program la linia de comandă, incluzând numele cu care a fost apelat programul.

Al doilea parametru, argv, este un pointer către conţinutul listei de parametri al căror număr este dat de argc. 

În mod normal, orice program care se execută corect va întoarce 0, şi o valoare diferită de 0 în cazul în care apar erori. 
    \end{minipage}
};

\node[fancytitle_orange, right=10pt] at (box.north west) {Funcția main};
\end{tikzpicture}

%---------------------------------------------------------------------------------

\begin{tikzpicture}
\node [mybox_orange] (box){%
    \begin{minipage}{0.3\textwidth}
\textbf
Tipul de date void are mai multe întrebuinţări.

Atunci când este folosit ca tip returnat de o funcţie, specifică faptul că funcţia nu întoarce nici o valoare. 

Exemplu:

\begin{lstlisting}[language=Haskell, numbers=none]
void print_nr(int number) {
  printf("Numarul este %d", number);
}
\end{lstlisting}

\textbf
Atunci când este folosit în declaraţia unei funcţii, void semnifică faptul că funcţia nu primeşte nici un parametru.

Exemplu:

\begin{lstlisting}[language=Haskell, numbers=none]
int init(void) {
  return 1;
}
\end{lstlisting}

    \end{minipage}
};

\node[fancytitle_orange, right=10pt] at (box.north west) {Tipul de date void};
\end{tikzpicture}





% %---------------------------------------------------------------------------------

% \begin{tikzpicture}
% \node [mybox_persianred] (box){%
%     \begin{minipage}{0.3\textwidth}\centering
% %     	{\centering\bf\color{persianred}
% 		\begin{lstlisting}[language=Racket]
% DA: (cons x L)        NU: (append (list x) L)
% 	                  NU: (append (cons x '()) L)
% DA: (if c vt vf)      NU: (if (equal? c #t) vt vf)
% DA: (null? L)         NU: (= (length L) 0)
% DA: (zero? x)         NU: (equal? x 0)
% DA: test              NU: (if test #t #f)
% DA: (or ceva1 ceva2)  NU: (if ceva1 #t ceva2)
% DA: (and ceva1 ceva2) NU: (if ceva1 ceva2 #f)
%         \end{lstlisting}
%     \end{minipage}
% };

% \node[fancytitle_persianred, right=10pt] at (box.north west) {AȘA  DA / AȘA NU};
% \end{tikzpicture}


% %---------------------------------------------------------------------------------

% % \begin{tikzpicture}
% % \node [mybox_code] (box){%
% %     \begin{minipage}{0.3\textwidth}
% % 		\begin{lstlisting}[language=Racket]

% %         \end{lstlisting}
% %     \end{minipage}
% % };

% % \node[fancytitle_code, right=10pt] at (box.north west) {Cum gandim un program};
% % \end{tikzpicture}

% %---------------------------------------------------------------------------------

% \begin{tikzpicture}
% \node [mybox_orange] (box){%
%     \begin{minipage}{0.3\textwidth}\small
%       \begin{enumerate}\itemsep.1ex
%       	\item După ce variabilă(e) fac recursivitatea? (ce variabilă(e) se schimbă de la un apel la altul?)
%         \item Care sunt condițiile de oprire în funcție de aceste variabile?(cazurile ``de bază'')
%         \item Ce se întâmplă când problema nu este încă elementară? (Obligatoriu aici cel puțin un apel recursiv)
%       \end{enumerate}
%     \end{minipage}
% };
% \node[fancytitle_orange, right=10pt] at (box.north west) {Programare cu funcții recursive};
% \end{tikzpicture}


% %---------------------------------------------------------------------------------

% \begin{tikzpicture}
% \node [mybox_neongreen] (box){%
%     \begin{minipage}{0.3\textwidth}\centering
% \href{http://docs.racket-lang.org/}{http://docs.racket-lang.org/}
%     \end{minipage}
% };

% \node[fancytitle_neongreen, right=10pt] at (box.north west) {Folositi cu incredere!};
% \end{tikzpicture}

%---------------------------------------------------------------------------------

\end{multicols*}
\end{document}
Contact GitHub API Training Shop Blog About
© 2016 GitHub, Inc. Terms Privacy Security Status Help
%%%%%%%%%%%%%%%%%%%%%%%%%%%%%%%%%%%%%%%%%%%%%%%%%%%%%%%
% MatPlotLib and Random Cheat Sheet
%
% Edited by Michelle Cristina de Sousa Baltazar
%
% http://matplotlib.org/api/pyplot_summary.html
% http://matplotlib.org/users/pyplot_tutorial.html
%
%%%%%%%%%%%%%%%%%%%%%%%%%%%%%%%%%%%%%%%%%%%%%%%%%%%%%%%

\documentclass[a4paper]{article}
\usepackage[landscape]{geometry}
\usepackage{url}
\usepackage{multicol}
\usepackage{amsmath}
\usepackage{amsfonts}
\usepackage{tikz}
\usetikzlibrary{decorations.pathmorphing}
\usepackage{amsmath,amssymb}
\usepackage{hyperref}

\usepackage{colortbl}
\usepackage{xcolor}
\usepackage{mathtools}
\usepackage{amsmath,amssymb}
\usepackage{enumitem}

% Mihnea
\usepackage{textcomp} % \textquotesingle: Racket: '(1 2 3)
\usepackage{couriers}
\usepackage{listings}
\lstset{
	numbers			= left,
	numberstyle		= \tiny,
	numbersep       = 5pt,
	captionpos		= b,
	breaklines		= true,
	basicstyle		= \ttfamily\footnotesize,
	tabsize			= 4,
	escapeinside	= {~}{~},
}
\lstdefinelanguage{Racket}{
  morekeywords=[1]{define, define-syntax, define-macro, lambda, define-stream, stream-lambda},
  morekeywords=[2]{begin, call-with-current-continuation, call/cc,
    call-with-input-file, call-with-output-file, case, cond,
    do, else, for-each, if,
    let*, let, let-syntax, letrec, letrec-syntax,
    let-values, let*-values,
    and, or, not, delay, force,
    quasiquote, quote, unquote, unquote-splicing,
    map, fold, syntax, syntax-rules, eval, environment },
  morekeywords=[3]{import, export},
  alsodigit=!\$\%&*+-./:<=>?@^_~,
  sensitive=true,
  morecomment=[l]{;},
  morecomment=[s]{\#|}{|\#},
  morestring=[b]",
  basicstyle=\footnotesize\ttfamily,
  keywordstyle=\color[rgb]{0,.3,.7},
  commentstyle=\color[rgb]{0.133,0.545,0.133},
  stringstyle={\color[rgb]{0.75,0.49,0.07}},
  upquote=true,
  breaklines=true,
  breakatwhitespace=true,
  literate=*{`}{{`}}{1}
}

\title{Racket}
\usepackage[brazilian]{babel}
\usepackage[utf8]{inputenc}

\advance\topmargin-1.0in
\advance\textheight3in
\advance\textwidth3in
\advance\oddsidemargin-1.5in
\advance\evensidemargin-1.5in
\parindent0pt
\parskip1pt
\newcommand{\hr}{\centerline{\rule{3.5in}{1pt}}}
%\colorbox[HTML]{e4e4e4}{\makebox[\textwidth-2\fboxsep][l]{texto}
\begin{document}

\begin{center}{\huge{\textbf{Haskell Intro CheatSheet}}}
% {\large Laborator}
\end{center}
\begin{multicols*}{3}

\tikzstyle{mybox} = [draw=black, fill=white, very thick,
    rectangle, rounded corners, inner sep=10pt, inner ysep=10pt]
\tikzstyle{fancytitle} =[fill=black, text=white, font=\bfseries]

% Mihnea
\tikzstyle{mybox_code} = [mybox, draw = orange, fill=sandybrown]
\tikzstyle{fancytitle_code} = [fancytitle, fill = orange]

\definecolor{almond}{rgb}{0.94, 0.87, 0.8}
\definecolor{apricot}{rgb}{0.98, 0.81, 0.69}
\definecolor{atomictangerine}{rgb}{1.0, 0.6, 0.4}
\definecolor{sandybrown}{rgb}{0.96, 0.64, 0.38}
\definecolor{buff}{rgb}{0.94, 0.86, 0.51}

\definecolor{persianred}{rgb}{0.8, 0.2, 0.2}
\definecolor{papayawhip}{rgb}{1.0, 0.94, 0.84}
\tikzstyle{mybox_persianred} = [mybox, draw = persianred, fill=papayawhip]
\tikzstyle{fancytitle_persianred} = [fancytitle, fill = persianred]

\definecolor{whitesmoke}{rgb}{0.96, 0.96, 0.96}
\definecolor{wenge}{rgb}{0.39, 0.33, 0.32}
\tikzstyle{mybox_blue} = [mybox, draw = wenge, fill=whitesmoke]
\tikzstyle{fancytitle_blue} = [fancytitle, fill = wenge]

\definecolor{cerise}{rgb}{0.87, 0.19, 0.39}
\definecolor{mistyrose}{rgb}{1.0, 0.89, 0.88}
\tikzstyle{mybox_cerise} = [mybox, draw = cerise, fill=mistyrose]
\tikzstyle{fancytitle_cerise} = [fancytitle, fill = cerise]

\definecolor{pinegreen}{rgb}{0.0, 0.47, 0.44}
\definecolor{bubbles}{rgb}{0.91, 1.0, 1.0}
\tikzstyle{mybox_pinegreen} = [mybox, draw = pinegreen, fill=bubbles]
\tikzstyle{fancytitle_pinegreen} = [fancytitle, fill = pinegreen]

\definecolor{cream}{rgb}{1.0, 0.99, 0.82}
\definecolor{mikadoyellow}{rgb}{1.0, 0.77, 0.05}
\tikzstyle{mybox_mikadoyellow} = [mybox, draw = mikadoyellow, fill=cream]
\tikzstyle{fancytitle_mikadoyellow} = [fancytitle, fill = mikadoyellow]

\definecolor{cornsilk}{rgb}{1.0, 0.97, 0.86}
\tikzstyle{mybox_orange} = [mybox, draw = orange, fill=cornsilk]
\tikzstyle{fancytitle_orange} = [fancytitle, fill = orange]

\definecolor{aliceblue}{rgb}{0.94, 0.97, 1.0}
\definecolor{seagreen}{rgb}{0.18, 0.55, 0.34}
\tikzstyle{mybox_seagreen} = [mybox, draw = seagreen, fill=aliceblue]
\tikzstyle{fancytitle_seagreen} = [fancytitle, fill = seagreen]

\definecolor{jazzberryjam}{rgb}{0.65, 0.04, 0.37}
\definecolor{almond}{rgb}{0.94, 0.87, 0.8}
\tikzstyle{mybox_jazzberryjam} = [mybox, draw = jazzberryjam, fill=almond]
\tikzstyle{fancytitle_jazzberryjam} = [fancytitle, fill = jazzberryjam]

\definecolor{amaranth}{rgb}{0.9, 0.17, 0.31}
\definecolor{bisque}{rgb}{1.0, 0.89, 0.77}
\tikzstyle{mybox_amaranth} = [mybox, draw = amaranth, fill=bisque]
\tikzstyle{fancytitle_amaranth} = [fancytitle, fill = amaranth]

\definecolor{carminered}{rgb}{1.0, 0.0, 0.22}
\definecolor{blanchedalmond}{rgb}{1.0, 0.92, 0.8}
\tikzstyle{mybox_carminered} = [mybox, draw = amaranth, fill=blanchedalmond]
\tikzstyle{fancytitle_carminered} = [fancytitle, fill = carminered]

\definecolor{midnightgreen}{rgb}{0.0, 0.29, 0.33}
\definecolor{lavendermist}{rgb}{0.9, 0.9, 0.98}
\tikzstyle{mybox_midnightgreen} = [mybox, draw = midnightgreen, fill=lavendermist]
\tikzstyle{fancytitle_midnightgreen} = [fancytitle, fill = midnightgreen]

\definecolor{indigo}{rgb}{0.29, 0.0, 0.51}
\definecolor{isabelline}{rgb}{0.96, 0.94, 0.93}
\tikzstyle{mybox_indigo} = [mybox, draw = indigo, fill=isabelline]
\tikzstyle{fancytitle_indigo} = [fancytitle, fill = indigo]

\definecolor{russet}{rgb}{0.5, 0.27, 0.11}
\definecolor{ivory}{rgb}{1.0, 1.0, 0.94}
\tikzstyle{mybox_russet} = [mybox, draw = russet, fill=ivory]
\tikzstyle{fancytitle_russet} = [fancytitle, fill = russet]

\definecolor{neongreen}{rgb}{0.22, 0.88, 0.08}
\definecolor{splashedwhite}{rgb}{1.0, 0.99, 1.0}
\tikzstyle{mybox_neongreen} = [mybox, draw = neongreen, fill=splashedwhite]
\tikzstyle{fancytitle_neongreen} = [fancytitle, fill = neongreen]

%---------------------------------------------------------------------------------

\begin{tikzpicture}
\node [mybox_persianred] (box){%
    \begin{minipage}{0.3\textwidth}

\begin{lstlisting}[language=Haskell, numbers=none]
5       :: Int
'H'     :: Char
"Hello" :: String
True    :: Bool
False   :: Bool
\end{lstlisting}
	\end{minipage}
};

\node[fancytitle_persianred, right=10pt] at (box.north west) {Tipuri de bază};
\end{tikzpicture}

\begin{tikzpicture}
\node [mybox_persianred] (box){%
    \begin{minipage}{0.3\textwidth}
    	{\centering\bf\small\color{seagreen} :t \\}

\begin{lstlisting}[language=Haskell, numbers=none]
> :t 42
42 :: Num a => a
\end{lstlisting}

\textbf{a} reprezintă o variabilă de tip,
restrictionată la toate tipurile numerice.

\begin{lstlisting}[language=Haskell, numbers=none]
> :t 42.0
42 :: Fractional a => a
\end{lstlisting}

In acest exemplu, \textbf{a} este
restrictionată la toate tipurile numerice fracționare
(e.g. \textbf{Float}, \textbf{Double}).

	\end{minipage}
};

\node[fancytitle_seagreen, right=10pt] at (box.north west) {Determinarea tipului unei expresii};
\end{tikzpicture}

\begin{tikzpicture}
\node [mybox_persianred] (box){%
    \begin{minipage}{0.3\textwidth}
    	{\centering\bf\small\color{seagreen} [] (:) \\}

\begin{lstlisting}[language=Haskell, numbers=none]
[]              -- lista vida
(:)             -- operatorul de adaugare
                -- la inceputul listei

1 : 3 : 5 : []  -- lista care contine 1, 3, 5
[1,3,5]         -- sintaxa echivalenta
\end{lstlisting}

	\end{minipage}
};

\node[fancytitle_jazzberryjam, right=10pt] at (box.north west) {Constructori liste};
\end{tikzpicture}

\begin{tikzpicture}
\node [mybox_amaranth] (box){%
    \begin{minipage}{0.3\textwidth}
{\centering\bf\small\color{amaranth} not \&\& $||$  \\}
		\begin{lstlisting}[language=Haskell, numbers=none]
not True                      False
not False                     True
True || False                 True
True && False                 False
        \end{lstlisting}
    \end{minipage}
};

\node[fancytitle_amaranth, right=10pt] at (box.north west) {Operatori logici};
\end{tikzpicture}

\begin{tikzpicture}
\node [mybox_orange] (box){%
    \begin{minipage}{0.3\textwidth}
    	{\centering\bf\small\color{pinegreen} (++) head tail last init take drop\\}
		\begin{lstlisting}[language=Haskell, numbers=none]
[1, 2] ++ [3, 4]                     [1, 2, 3, 4]

head [1, 2, 3, 4]                    1
tail [1 2 3 4]                       [2, 3, 4]

last [1, 2, 3, 4]                    4
init [1, 2, 3, 4]                    [1, 2, 3]

take 2 [1, 2, 3, 4]                  [1, 2]
take 2 "HelloWorld"                  "He"

drop 2 [1, 2, 3, 4]                  [3, 4]

null []                              True
null [1, 2, 3]                       False
\end{lstlisting}
    \end{minipage}
};

\node[fancytitle_orange, right=10pt] at (box.north west) {Operatori pe liste};
\end{tikzpicture}

\begin{tikzpicture}
\node [mybox_orange] (box){%
    \begin{minipage}{0.3\textwidth}
	{\centering\bf\small\color{pinegreen} length elem reverse\\}
		\begin{lstlisting}[language=Haskell, numbers=none]
length [1, 2, 3, 4]                  4

elem 3 [1, 2, 3, 4]                  True
elem 5 [1, 2, 3, 4]                  False

reverse [1, 2, 3, 4]                 [4, 3, 2, 1]
        \end{lstlisting}
    \end{minipage}
};

\node[fancytitle_orange, right=10pt] at (box.north west) {Alte operații};
\end{tikzpicture}

%---------------------------------------------------------------------------------

\begin{tikzpicture}
\node [mybox_orange] (box){%
    \begin{minipage}{0.3\textwidth}
% {\centering\bf\small\color{midnightgreen}    Tupluri}

Spre deosebire de liste, tuplurile au un număr fix de elemente, iar acestea
pot avea tipuri diferite.
\begin{lstlisting}[language=Haskell, numbers=none]
import Data.Tuple

("Hello", True) :: (String, Bool)
(1,2,3)         :: (Integer, Integer, Integer)

fst ("Hello", True)     "Hello"
snd ("Hello", True)     True
swap ("Hello", True)    (True, "Hello")
\end{lstlisting}

    \end{minipage}
};

\node[fancytitle_orange, right=10pt] at (box.north west) {Tupluri};
\end{tikzpicture}


\begin{tikzpicture}
\node [mybox_midnightgreen] (box){%
    \begin{minipage}{0.3\textwidth}
{\centering \bf\small\color{midnightgreen} \textbackslash arg1 arg2 $\rightarrow$ corp\\}

\begin{lstlisting}[language=Haskell, numbers=none]
\x -> x                       functia identitate
(\x y -> x + y) 1 2           3
let f = \x y -> x + y         legare la un nume
(f 1 2)                       3
\end{lstlisting}
\end{minipage}
};

\node[fancytitle_midnightgreen, right=10pt] at (box.north west) {Funcții anonime (lambda)};
\end{tikzpicture}

%---------------------------------------------------------------------------------

\begin{tikzpicture}
\node [mybox_indigo] (box){%
    \begin{minipage}{0.3\textwidth}

\begin{lstlisting}[language=Haskell, numbers=none]
-- if .. then .. else
factorial x =
    if x < 1 then 1 else x * factorial (x - 1)

-- guards
factorial x
    | x < 1 = 1
    | otherwise = x * factorial (x - 1)

-- case .. of
factorial x = case x < 1 of
    True -> 1
    _    -> x * factorial (x - 1)

-- pattern matching
factorial 0 = 1
factorial x = x * factorial (x - 1)

\end{lstlisting}
\end{minipage}
};

\node[fancytitle_indigo, right=10pt] at (box.north west) {Definire functii};
\end{tikzpicture}

% %---------------------------------------------------------------------------------

\begin{tikzpicture}
\node [mybox_indigo] (box){
    \begin{minipage}{0.3\textwidth}

In Haskell funcțiile sunt, by default, in forma curry.

\begin{lstlisting}[language=Haskell, numbers=none]

:t (+)
(+) :: Num a => a -> a -> a

:t (+ 1)
(+ 1) :: Num a => a -> a

\end{lstlisting}
\end{minipage}
};

\node[fancytitle_indigo, right=10pt] at (box.north west) {Curry};
\end{tikzpicture}

\begin{tikzpicture}
\node [mybox_indigo] (box){
    \begin{minipage}{0.3\textwidth}
\begin{lstlisting}[language=Haskell, numbers=none]

map :: (a -> b) -> [a] -> [b]
filter :: (a -> Bool) -> [a] -> [a]
foldl :: (a -> b -> a) -> a -> [b] -> a
zip   :: [a] -> [b] -> [(a, b)]

map (+ 2) [1, 2, 3]         [3, 4, 5]

filter odd [1, 2, 3, 4]     [1 3]

foldl (+) 0 [1, 2, 3, 4]    10
foldl (-) 0 [1, 2]          -3   (0 - 1) - 2
foldr (-) 0 [1, 2]          -1   1 - (2 - 0)

zip [1,2] [3, 4]            [(1,3), (2,4)]

\end{lstlisting}
\end{minipage}
};

\node[fancytitle_indigo, right=10pt] at (box.north west) {Functionale uzuale};
\end{tikzpicture}

% %---------------------------------------------------------------------------------

% \begin{tikzpicture}
% \node [mybox_persianred] (box){%
%     \begin{minipage}{0.3\textwidth}\centering
% %     	{\centering\bf\color{persianred}
% 		\begin{lstlisting}[language=Racket]
% DA: (cons x L)        NU: (append (list x) L)
% 	                  NU: (append (cons x '()) L)
% DA: (if c vt vf)      NU: (if (equal? c #t) vt vf)
% DA: (null? L)         NU: (= (length L) 0)
% DA: (zero? x)         NU: (equal? x 0)
% DA: test              NU: (if test #t #f)
% DA: (or ceva1 ceva2)  NU: (if ceva1 #t ceva2)
% DA: (and ceva1 ceva2) NU: (if ceva1 ceva2 #f)
%         \end{lstlisting}
%     \end{minipage}
% };

% \node[fancytitle_persianred, right=10pt] at (box.north west) {AȘA  DA / AȘA NU};
% \end{tikzpicture}


% %---------------------------------------------------------------------------------

% % \begin{tikzpicture}
% % \node [mybox_code] (box){%
% %     \begin{minipage}{0.3\textwidth}
% % 		\begin{lstlisting}[language=Racket]

% %         \end{lstlisting}
% %     \end{minipage}
% % };

% % \node[fancytitle_code, right=10pt] at (box.north west) {Cum gandim un program};
% % \end{tikzpicture}

% %---------------------------------------------------------------------------------

% \begin{tikzpicture}
% \node [mybox_orange] (box){%
%     \begin{minipage}{0.3\textwidth}\small
%       \begin{enumerate}\itemsep.1ex
%       	\item După ce variabilă(e) fac recursivitatea? (ce variabilă(e) se schimbă de la un apel la altul?)
%         \item Care sunt condițiile de oprire în funcție de aceste variabile?(cazurile ``de bază'')
%         \item Ce se întâmplă când problema nu este încă elementară? (Obligatoriu aici cel puțin un apel recursiv)
%       \end{enumerate}
%     \end{minipage}
% };
% \node[fancytitle_orange, right=10pt] at (box.north west) {Programare cu funcții recursive};
% \end{tikzpicture}


% %---------------------------------------------------------------------------------

% \begin{tikzpicture}
% \node [mybox_neongreen] (box){%
%     \begin{minipage}{0.3\textwidth}\centering
% \href{http://docs.racket-lang.org/}{http://docs.racket-lang.org/}
%     \end{minipage}
% };

% \node[fancytitle_neongreen, right=10pt] at (box.north west) {Folositi cu incredere!};
% \end{tikzpicture}

%---------------------------------------------------------------------------------

\end{multicols*}
\end{document}
Contact GitHub API Training Shop Blog About
© 2016 GitHub, Inc. Terms Privacy Security Status Help
