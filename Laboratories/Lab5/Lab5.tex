%%%%%%%%%%%%%%%%%%%%%%%%%%%%%%%%%%%%%%%%%%%%%%%%%%%%%%%
% MatPlotLib and Random Cheat Sheet
%
% Edited by Michelle Cristina de Sousa Baltazar
%
% http://matplotlib.org/api/pyplot_summary.html
% http://matplotlib.org/users/pyplot_tutorial.html
%
%%%%%%%%%%%%%%%%%%%%%%%%%%%%%%%%%%%%%%%%%%%%%%%%%%%%%%%

\documentclass[a4paper]{article}
\usepackage[landscape]{geometry}
\usepackage{url}
\usepackage{multicol}
\usepackage{amsmath}
\usepackage{amsfonts}
\usepackage{tikz}
\usetikzlibrary{decorations.pathmorphing}
\usepackage{amsmath,amssymb}
\usepackage{hyperref}

\usepackage{colortbl}
\usepackage{xcolor}
\usepackage{mathtools}
\usepackage{amsmath,amssymb}
\usepackage{enumitem}

% Mihnea
\usepackage{textcomp} % \textquotesingle: Racket: '(1 2 3)
\usepackage{couriers}
\usepackage{listings}
\lstset{
	numbers			= left,
	numberstyle		= \tiny,
	numbersep       = 5pt,
	captionpos		= b,
	breaklines		= true,
	basicstyle		= \ttfamily\footnotesize,
	tabsize			= 4,
	escapeinside	= {~}{~},
}
\lstdefinelanguage{Racket}{
  morekeywords=[1]{define, define-syntax, define-macro, lambda, define-stream, stream-lambda},
  morekeywords=[2]{begin, call-with-current-continuation, call/cc,
    call-with-input-file, call-with-output-file, case, cond,
    do, else, for-each, if,
    let*, let, let-syntax, letrec, letrec-syntax,
    let-values, let*-values,
    and, or, not, delay, force,
    quasiquote, quote, unquote, unquote-splicing,
    map, fold, syntax, syntax-rules, eval, environment },
  morekeywords=[3]{import, export},
  alsodigit=!\$\%&*+-./:<=>?@^_~,
  sensitive=true,
  morecomment=[l]{;},
  morecomment=[s]{\#|}{|\#},
  morestring=[b]",
  basicstyle=\footnotesize\ttfamily,
  keywordstyle=\color[rgb]{0,.3,.7},
  commentstyle=\color[rgb]{0.133,0.545,0.133},
  stringstyle={\color[rgb]{0.75,0.49,0.07}},
  upquote=true,
  breaklines=true,
  breakatwhitespace=true,
  literate=*{`}{{`}}{1}
}

\title{Racket}
\usepackage[brazilian]{babel}
\usepackage[utf8]{inputenc}

\advance\topmargin-1.0in
\advance\textheight3in
\advance\textwidth3in
\advance\oddsidemargin-1.5in
\advance\evensidemargin-1.5in
\parindent0pt
\parskip1pt
\newcommand{\hr}{\centerline{\rule{3.5in}{1pt}}}
%\colorbox[HTML]{e4e4e4}{\makebox[\textwidth-2\fboxsep][l]{texto}
\begin{document}

\begin{center}{\huge{\textbf{Tablouri. Particularizare - vectori}}}
% {\large Laborator}
\end{center}
\begin{multicols*}{3}

\tikzstyle{mybox} = [draw=black, fill=white, very thick,
    rectangle, rounded corners, inner sep=10pt, inner ysep=10pt]
\tikzstyle{fancytitle} =[fill=black, text=white, font=\bfseries]

% Mihnea
\tikzstyle{mybox_code} = [mybox, draw = orange, fill=sandybrown]
\tikzstyle{fancytitle_code} = [fancytitle, fill = orange]

\definecolor{almond}{rgb}{0.94, 0.87, 0.8}
\definecolor{apricot}{rgb}{0.98, 0.81, 0.69}
\definecolor{atomictangerine}{rgb}{1.0, 0.6, 0.4}
\definecolor{sandybrown}{rgb}{0.96, 0.64, 0.38}
\definecolor{buff}{rgb}{0.94, 0.86, 0.51}

\definecolor{persianred}{rgb}{0.8, 0.2, 0.2}
\definecolor{papayawhip}{rgb}{1.0, 0.94, 0.84}
\tikzstyle{mybox_persianred} = [mybox, draw = persianred, fill=papayawhip]
\tikzstyle{fancytitle_persianred} = [fancytitle, fill = persianred]

\definecolor{whitesmoke}{rgb}{0.96, 0.96, 0.96}
\definecolor{wenge}{rgb}{0.39, 0.33, 0.32}
\tikzstyle{mybox_blue} = [mybox, draw = wenge, fill=whitesmoke]
\tikzstyle{fancytitle_blue} = [fancytitle, fill = wenge]

\definecolor{cerise}{rgb}{0.87, 0.19, 0.39}
\definecolor{mistyrose}{rgb}{1.0, 0.89, 0.88}
\tikzstyle{mybox_cerise} = [mybox, draw = cerise, fill=mistyrose]
\tikzstyle{fancytitle_cerise} = [fancytitle, fill = cerise]

\definecolor{pinegreen}{rgb}{0.0, 0.47, 0.44}
\definecolor{bubbles}{rgb}{0.91, 1.0, 1.0}
\tikzstyle{mybox_pinegreen} = [mybox, draw = pinegreen, fill=bubbles]
\tikzstyle{fancytitle_pinegreen} = [fancytitle, fill = pinegreen]

\definecolor{cream}{rgb}{1.0, 0.99, 0.82}
\definecolor{mikadoyellow}{rgb}{1.0, 0.77, 0.05}
\tikzstyle{mybox_mikadoyellow} = [mybox, draw = mikadoyellow, fill=cream]
\tikzstyle{fancytitle_mikadoyellow} = [fancytitle, fill = mikadoyellow]

\definecolor{cornsilk}{rgb}{1.0, 0.97, 0.86}
\tikzstyle{mybox_orange} = [mybox, draw = orange, fill=cornsilk]
\tikzstyle{fancytitle_orange} = [fancytitle, fill = orange]

\definecolor{aliceblue}{rgb}{0.94, 0.97, 1.0}
\definecolor{seagreen}{rgb}{0.18, 0.55, 0.34}
\tikzstyle{mybox_seagreen} = [mybox, draw = seagreen, fill=aliceblue]
\tikzstyle{fancytitle_seagreen} = [fancytitle, fill = seagreen]

\definecolor{jazzberryjam}{rgb}{0.65, 0.04, 0.37}
\definecolor{almond}{rgb}{0.94, 0.87, 0.8}
\tikzstyle{mybox_jazzberryjam} = [mybox, draw = jazzberryjam, fill=almond]
\tikzstyle{fancytitle_jazzberryjam} = [fancytitle, fill = jazzberryjam]

\definecolor{amaranth}{rgb}{0.9, 0.17, 0.31}
\definecolor{bisque}{rgb}{1.0, 0.89, 0.77}
\tikzstyle{mybox_amaranth} = [mybox, draw = amaranth, fill=bisque]
\tikzstyle{fancytitle_amaranth} = [fancytitle, fill = amaranth]

\definecolor{carminered}{rgb}{1.0, 0.0, 0.22}
\definecolor{blanchedalmond}{rgb}{1.0, 0.92, 0.8}
\tikzstyle{mybox_carminered} = [mybox, draw = amaranth, fill=blanchedalmond]
\tikzstyle{fancytitle_carminered} = [fancytitle, fill = carminered]

\definecolor{midnightgreen}{rgb}{0.0, 0.29, 0.33}
\definecolor{lavendermist}{rgb}{0.9, 0.9, 0.98}
\tikzstyle{mybox_midnightgreen} = [mybox, draw = midnightgreen, fill=lavendermist]
\tikzstyle{fancytitle_midnightgreen} = [fancytitle, fill = midnightgreen]

\definecolor{indigo}{rgb}{0.29, 0.0, 0.51}
\definecolor{isabelline}{rgb}{0.96, 0.94, 0.93}
\tikzstyle{mybox_indigo} = [mybox, draw = indigo, fill=isabelline]
\tikzstyle{fancytitle_indigo} = [fancytitle, fill = indigo]

\definecolor{russet}{rgb}{0.5, 0.27, 0.11}
\definecolor{ivory}{rgb}{1.0, 1.0, 0.94}
\tikzstyle{mybox_russet} = [mybox, draw = russet, fill=ivory]
\tikzstyle{fancytitle_russet} = [fancytitle, fill = russet]

\definecolor{neongreen}{rgb}{0.22, 0.88, 0.08}
\definecolor{splashedwhite}{rgb}{1.0, 0.99, 1.0}
\tikzstyle{mybox_neongreen} = [mybox, draw = neongreen, fill=splashedwhite]
\tikzstyle{fancytitle_neongreen} = [fancytitle, fill = neongreen]

%---------------------------------------------------------------------------------

\begin{tikzpicture}
\node [mybox_persianred] (box){%
    \begin{minipage}{0.3\textwidth}
\textbf
Printr-un vector se înţelege o colecţie liniară şi omogenă de date. 

Un vector este liniar pentru că datele(elementele) pot fi accesate în mod unic printr-un index.

Un vector este, de asemenea, omogen, pentru că toate elementele sunt de acelaşi tip. În limbajul C, indexul este un număr întreg pozitiv şi indexarea se face începând cu 0.

Declaraţia unei variabile de tip vector se face în felul următor:

\begin{lstlisting}[language=Haskell, numbers=none]
<tip_elemente> <nume_vector>[<dimensiune>];

int a[100];
float vect[50];
 
#define MAX 100
...
unsigned long numbers[MAX]


int a[3] = {1, 5, 6};                     // Toate cele 3 elemente sunt initializate 
float num[] = {1.5, 2.3, 0.2, -1.3};      // Compilatorul determina dimensiunea - 4 - a vectorului
unsigned short vect[1000] = {0, 2, 4, 6}; // Sunt initializate doar primele 4 elemente

char string[100] = {97}; //Sirul va fi initializat cu 97 (caracterul 'a') pe prima pozitie si 99 de 0
int v[100] = {0};        // Vectorul va fi umplut cu 0.

\end{lstlisting}

\textbf
Elementele neiniţializate pot avea valori oarecare.
Elementele se accesează prin expresii de forma:
\begin{lstlisting}[language=Haskell, numbers=none]
<nume_vector>[<indice>].
\end{lstlisting}
\textbf
De exemplu, putem avea:
\begin{lstlisting}[language=Haskell, numbers=none]
char vect[100];
vect[0] = 1;
vect[5] = 10;
 
int i = 90;
vect[i] = 15;
vect[i + 1] = 20;
\end{lstlisting}
\textbf 
La alocarea unui vector, compilatorul nu efectuează nici un fel de iniţializare şi nu furnizează nici un mesaj de eroare dacă un element este folosit înainte de a fi iniţializat. Un program corect va iniţializa, în orice caz, fiecare element înainte de a-l folosi.




	\end{minipage}
};



\node[fancytitle_persianred, right=10pt] at (box.north west) {Vectori};
\end{tikzpicture}

\begin{tikzpicture}
\node [mybox_persianred] (box){%
    \begin{minipage}{0.3\textwidth}
    
\textbf
Citirea unui vector de intregi de la tastatura:
\begin{lstlisting}[language=Haskell, numbers=none]
int main() 
{
  int a[100], n, i; /* vectorul a are maxim 100 de intregi */
 
  scanf("%d", &n); /* citeste nr de elemente vector */
 
  for (i = 0; i < n; i++) {
    scanf("%d", &a[i]); /* citire elemente vector */
  }
 
  for (i = 0; i < n; i++) {
    printf("%d ", a[i]); /* scrie elemente vector */
  }
 
  return 0;
}
\end{lstlisting}
\textbf
Generarea unui vector cu primele n numere Fibonacci:
\begin{lstlisting}[language=Haskell, numbers=none]
#include <stdio.h>
int main() 
{
  long fib[100] = {1, 1};
  int n, i;
 
  printf("n = "); 
  scanf("%d", &n);
 
  for (i = 2; i < n; i++) {
    fib[i] = fib[i - 1] + fib[i - 2];
  }
  for (i = 0; i < n; i++) {
    printf("%ld ", fib[i]);
  }
 
  return 0;
}
\end{lstlisting}
\textbf
Definiţi dimensiunile prin constante şi folosiţi-le pe acestea în locul tastării explicite a valorilor în codul sursă.
\begin{lstlisting}[language=Haskell, numbers=none]
#define MAX   100
 
int vect[MAX];
\end{lstlisting}
\textbf
va fi de preferat în locul lui:
\begin{lstlisting}[language=Haskell, numbers=none]
int vect[100];
\end{lstlisting}

	\end{minipage}
};

\node[fancytitle_seagreen, right=10pt] at (box.north west) {Exemple de programe};
\end{tikzpicture}

\begin{tikzpicture}
\node [mybox_persianred] (box){%
    \begin{minipage}{0.3\textwidth}
\textbf
Cautarea secventiala:
\begin{lstlisting}[language=Haskell, numbers=none]
#define MAX 100
 ...
 int v[MAX], x, i;
 /* initializari */
...
 int found = 0;
for (i = 0; i < MAX; i++) {
  if (x == v[i]) {
    found = 1;
    break;
  }
}
 if (found) {
   printf("Valoarea %d a fost gasita in vector\n", x);
} else { 
   printf("Valoarea %d nu a fost gasita in vector\n", x);
}
\end{lstlisting}
\textbf
Cautarea binara iterativa:
\begin{lstlisting}[language=Haskell, numbers=none]
/ cauta elementul x in vectorul sortat v, intre pozitiile 0 si n-1  si returneaza pozitia gasita sau -1
int binary_search(int n, int v[NMAX], int x) {
  int low = 0, high = n - 1;
 
  while (low <= high) {
    int middle = low + (high - low) / 2;
 
    if (v[middle] == x) {
      // Am gasit elementul, returnam pozitia sa
      return middle;
    }
 
    if (v[middle] < x) {
      // Elementul cautat este mai mare decat cel curent, ne mutam in jumatatea
      // cu elemente mai mari
      low = middle + 1;
    } else {
      // Elementul cautat este mai mic decat cel curent, ne mutam in jumatatea
      // cu elemente mai mici
      high = middle - 1;
    }
  }
 
  // Elementul nu a fost gasit
  return -1;
}
\end{lstlisting}
	\end{minipage}
};

\node[fancytitle_jazzberryjam, right=10pt] at (box.north west) {Cautari};
\end{tikzpicture}



% %---------------------------------------------------------------------------------


% %---------------------------------------------------------------------------------

% \begin{tikzpicture}
% \node [mybox_persianred] (box){%
%     \begin{minipage}{0.3\textwidth}\centering
% %     	{\centering\bf\color{persianred}
% 		\begin{lstlisting}[language=Racket]
% DA: (cons x L)        NU: (append (list x) L)
% 	                  NU: (append (cons x '()) L)
% DA: (if c vt vf)      NU: (if (equal? c #t) vt vf)
% DA: (null? L)         NU: (= (length L) 0)
% DA: (zero? x)         NU: (equal? x 0)
% DA: test              NU: (if test #t #f)
% DA: (or ceva1 ceva2)  NU: (if ceva1 #t ceva2)
% DA: (and ceva1 ceva2) NU: (if ceva1 ceva2 #f)
%         \end{lstlisting}
%     \end{minipage}
% };

% \node[fancytitle_persianred, right=10pt] at (box.north west) {AȘA  DA / AȘA NU};
% \end{tikzpicture}


% %---------------------------------------------------------------------------------

% % \begin{tikzpicture}
% % \node [mybox_code] (box){%
% %     \begin{minipage}{0.3\textwidth}
% % 		\begin{lstlisting}[language=Racket]

% %         \end{lstlisting}
% %     \end{minipage}
% % };

% % \node[fancytitle_code, right=10pt] at (box.north west) {Cum gandim un program};
% % \end{tikzpicture}

% %---------------------------------------------------------------------------------

% \begin{tikzpicture}
% \node [mybox_orange] (box){%
%     \begin{minipage}{0.3\textwidth}\small
%       \begin{enumerate}\itemsep.1ex
%       	\item După ce variabilă(e) fac recursivitatea? (ce variabilă(e) se schimbă de la un apel la altul?)
%         \item Care sunt condițiile de oprire în funcție de aceste variabile?(cazurile ``de bază'')
%         \item Ce se întâmplă când problema nu este încă elementară? (Obligatoriu aici cel puțin un apel recursiv)
%       \end{enumerate}
%     \end{minipage}
% };
% \node[fancytitle_orange, right=10pt] at (box.north west) {Programare cu funcții recursive};
% \end{tikzpicture}


% %---------------------------------------------------------------------------------

% \begin{tikzpicture}
% \node [mybox_neongreen] (box){%
%     \begin{minipage}{0.3\textwidth}\centering
% \href{http://docs.racket-lang.org/}{http://docs.racket-lang.org/}
%     \end{minipage}
% };

% \node[fancytitle_neongreen, right=10pt] at (box.north west) {Folositi cu incredere!};
% \end{tikzpicture}

%---------------------------------------------------------------------------------

\end{multicols*}
\end{document}
Contact GitHub API Training Shop Blog About
© 2016 GitHub, Inc. Terms Privacy Security Status Help
